% $Id$

The {\tt Date} class provides a set of functions for manipulating dates.
These include setting and retrieving dates, incrementing and decrementing 
dates by a specified time interval, taking the difference of two dates,
determining whether one date is later than another, and computing the
day of year of a given date.
   
The {\tt Date} class contains attributes representing year, month and day 
quantities and a time of day.  It also contains a calendar which 
stores, for a given year, such quantities as the number of days per 
month and per year.  Gregorian and no-leap year calendars are currently 
supported.  

The algorithm to convert from Gregorian to Julian days is from 
Henry F. Fliegel and Thomas C. Van Flandern, in Communications of 
the ACM (CACM, volume 11, number 10, October 1968, p.657).  Julian 
day refers to the number of days since a reference day.  For the 
algorithm used, this reference day is November 24, -4713 in the Gregorian 
calendar.  The algorithm is valid through all future dates, assuming 
standard corrections are applied (at 4 years, 100 years,
and 400 years).


