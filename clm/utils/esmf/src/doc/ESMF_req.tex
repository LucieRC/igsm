% $Id$
\section{Requirements}

The Earth System Modeling Framework (ESMF) will consist of an interface
specification and a reference implementation.  The ESMF is intended to 
facilitate coupling of model components and to support lower-level
non-scientific tasks on high performance computer platforms.

The ESMF will support predictive models of the atmosphere, ocean, land and
sea ice and data assimilation systems.

More detailed functional requirements will be prepared as the project 
proceeds.

\subsection{Functional Requirements}

\subsubsection{Coupling Superstructure}

The ESMF will provide a coupling mechanism that performs regridding,
interpolation and communication of gridded, distributed data.  The 
data may represent multiple fields or a single field, may be in the
same or different executables, may be in code segments executing 
synchronously or asynchronously, and may be distributed among nodes
and/or partitioned among multiple threads.

\subsubsection{Parallel Support Layer}

The ESMF will provide the software necessary to support the data
decomposition and communication requirements of the individual components
and will do so in a way that is consistent with the superstructure.
This layer will include
tools for describing a wide variety of grids and decompositions,
and for performing high-level collective manipulations of fields defined
on those grids.  The software for specifying decompositions should support
default decompositions and dynamic load balancing.

\subsubsection{Utility Infrastructure}

The ESMF will include general purpose utility routines for use by both 
the coupling mechanism and application codes.  These include
performance profiling, time management and error handling.

The ESMF will support I/O of self-describing data in netCDF, HDF, 
binary, GRIB and BUFR data formmats.  Others such as the EOS HDF and ODF 
data formats are desired but not required.  The I/O utilities will 
have generic interfaces for ease of use, and will be high-performance.

\subsubsection{Evaluation Suite}

The ESMF will be distributed with a suite of representative components
that will demonstrate its usage.

\subsection{General Computational Requirements}

\subsubsection{Performance Portability}

Portability and computational efficiency over a wide range of platforms
is essential.  ESMF must be supported on the following platforms:
\begin{itemize}
\item Cray T3E (initial ESS testbed)
\item IBM SP
\item SGI Origin 2000
\item Compaq ES40
\item PC Linux platforms (including cluster)
\end{itemize}

Optimized performance on scalar architectures with moderate numbers of 
processors (100-1000) is the highest priority. 

Because many of the platforms listed above support multiple 
layers of parallelism (e.g. MPI, OpenMP),  the framework
must support message-based and thread-based parallelism and
hybrid combinations of the two approaches.

The framework will not increase the execution time of an existing code
written without the framework by more than 10\%.

The framework must include performance measurement tools. 

\subsubsection{Language}

ESMF utility and coupling software must have both Fortran 90 and
C/C++ bindings.

\subsubsection{Grids}

ESMF must support the coupling of components that are discretized on:
\begin{itemize}
\item logically rectangular grids
\item reduced (cut-out) and regional grids
\item unstructured grids (e.g., land grids, observations)
\item phase space grids (e.g., spectral, Fourier)
\item nested and adaptive grids
\item cubed sphere and icosahedral grids
\end{itemize}

In addition we require support for describing masked regions and
halo regions.

\subsubsection{Runtime Configurability}

ESMF must allow domain decomposition and the number of processors and/or
nodes to be configurable at runtime.  Other types of runtime configurability
may be permitted if acceptable performance is sustained.

\subsubsection{Fault Tolerance}

Error reporting must be handled consistently and with ample information
relayed to the user.  In situations where the components fail,
the ESMF will have a mechanism to detect the failure and shut down the
entire application.

\subsection{Design, Implementation and Maintenance Requirements}

\subsubsection{Flexibility and Extensibility}

Layers of the framework will be designed to survive restructuring of
other parts of the framework and user-supplied components.  For example,
the coupling layers should be able to adapt to different implementations
and data structures of component models.

\subsubsection{Ease of Adoption}

It must be straightforward to integrate ESMF into an application that
is reasonably modular.  We adopt as a goal that such applications should
need to modify no more than 2\% of their source code to utilize the coupling
features of ESMF.

\subsubsection{Ongoing Support}

The ESMF must be maintained as a long-term commitment by at least one
institution.  This maintenance must extend beyond adaptation to the 
computational environment, and must include an ongoing research component
dedicated to increasing the performance, flexibility and functionality of
the software.










