\subsubsection{General Usage}
The basic sequence for use of the library is to first call {\tt ESMF\_TimerInit}, either
once to initialize the behavior of all timers, or multiple times to initialize individual
timers (see Appendix A for further descriptions of any individual function).

Next, the {\tt ESMF\_TimerStart} function is called before entering a code section of interest.  When
the section completes, the {\tt ESMF\_TimerStop} function is called.  This may be repeated 
for different timers and/or for the same timer.

Lastly, the information collected by the timer is output.  This is done via a call to 
{\tt ESMF\_TimerPrint}.  This function prints a timer or all timers to a {\tt ESMF\_Log} 
object.  If the log object is not given, it prints to the standard application log 
(see {\tt ESMF\_App for details}.  Otherwise any user created log object may be specified.
  The call also takes either the {\tt ESMF\_ALL\_TIMERS} handle, in which event it prints all
timers, or the name of a specific timer, in which event it prints that timer only.

\subsubsection{Code Example}
The following code segment shows a basic use of the timer library.

\begin{verbatim}
        program TimerEx
        Use ESMF_AppMod
        Use ESMF_TimerMod
        implicit none
        type(ESMF_App) :: app
        character*7 timerName
        integer :: i

! An application must be initialized for the library to be used.
        app = ESMF_AppNew()

! This call initializes all default timers.  It enables the tracking/reporting
! of user and system time, wall-clock time, and floating point instructions.  
! The list is ended with a 0 to signal the end of the option.
        call ESMF_TimerInit(ESMF_ALL_TIMERS,
     &       ESMF_USRSYS, 1,  ESMF_WALL, 1,
     &       ESMF_FP_INSTR, 1, 0)

! Create an outer-loop timer to time the total execution.  This timer demonstrates
! that the start/stop pairs may be nested, provided that the nesting uses different
! timer names.
        call ESMF_TimerStart("End2End")

! Create a timer for a subroutine that is called once.
! Start the timer.
        call ESMF_TimerStart("Task1")

! Call the subroutine.
        call task1()

! End the timer.
        call ESMF_TimerStop("Task1")

! Create a timer for a task that is called several times.  This timer
! will contain both the sum of the times and stats.
        do i=1,3
          call ESMF_TimerStart("Task2")
          call task2()
          call ESMF_TimerStop("Task2")
        enddo

! Stop the outer-loop timer.
        call ESMF_TimerStop("End2End")

! Create a report for all three timers to the standard log-file.
        call ESMF_TimerPrint(name=ESMF_ALL_TIMERS)

! The application must be destroyed to finalize the program.
        call ESMF_AppDelete(app)
        end program

\end{verbatim}

\subsubsection{Example Output}

The following is an example output file from the library:

\begin{verbatim}
Wallclock timer cost est.:   24.000 usec per start/stop pair
CPU timer cost est.:         24.000 usec per start/stop pair
CPU accumulation interval is 0.01 seconds
Clock speed is 195 MHz

Stats for thread 0:
Name        Called   Usr     Sys     Usr+Sys   Wallclock Max      Min     FP instr
End2End     1        6.990   0.010   7.000     7.145     7.145    7.145   1.049e+08
  Task1     1        1.750   0.000   1.750     1.880     1.880    1.880   2.621e+07
  Task2     3        5.240   0.000   5.240     5.263     1.755    1.754   7.864e+07


TIMER OVERHEAD (wallclock seconds) =     0.002252

\end{verbatim}

Each timer is listed, with the requested stats from each.  Notice that the {\tt Task2} timer was
called three times.  The totals for that timer are listed, as well as the minimum and maximum
times for any single call.  The timer gives an estimate as to the overhead of the library.  The number
of floating point instructions appears in the right column.
