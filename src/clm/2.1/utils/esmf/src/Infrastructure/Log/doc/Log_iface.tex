% $Id$
\subsection{F90 Interface}
\subsubsection{Types}

{\bf MFM\_LogHandle}

This log handle is returned from the initialization call.  It is used to
identify which log a given command is to use.  This type may only contain an
integer handle, but is cast as a Fortran type for {\tt type } safety.

\subsubsection{Log Levels}

Each message that is output via the log will have an associated logging
level.  This level is used to categorize what type of message is being output
and what the priority/category of the message is.
The logging levels will be defined as a list of integer parameters:

\begin{itemize}
\item{\bf MFM\_LOGLEVEL\_INFO}

Marks a message that would be useful for someone debugging the code to have.

\item{\bf MFM\_LOGLEVEL\_ERROR}

Reports an error condition.

\item{\bf MFM\_LOGLEVEL\_TIMER}

Used to write information related to code profiling or timing.

\end{itemize}

\subsubsection{Log States}

The program may be run at several different logging states.  Each state will
allow the messages at various logging levels to be output. 

\begin{itemize}
\item{\bf MFM\_LOGSTATE\_QUIET}

Suppresses the output of all message types.  All calls to the log will
return immediately after verifying this flag.  This will be the fastest mode of
operation, the only faster being to turn the library off via the compile flags.

\item{\bf MFM\_LOGSTATE\_NORMAL}

This option will print only the types of messages that you would want to have
under normal operations.

\item{\bf MFM\_LOGSTATE\_TIMER}

Prints out only profiling/timing information.

\item{\bf MFM\_LOGSTATE\_VERBOSE}

Outputs every type of message.

\end{itemize}

\subsubsection{Output Matrix}

A matrix will determine which types of messages are output in each state.
\begin{tabular}{|p{1.5in}|p{3.33in}|} \hline
  & {\bf Log Level} \\ \hline
\end{tabular}
\begin{tabular}{|p{1.5in}|p{1.0in}|p{1.0in}|p{1.0in}|} \hline
{\bf Log State}   & {\bf INFO }& {\bf ERROR } & {\bf TIMER} \\ \hline
{\bf QUIET}       & off & off & off \\ \hline
{\bf NORMAL}      & on & on & off \\ \hline
{\bf TIMER}       & off & off & on \\ \hline
{\bf VERBOSE}     & on & on  & on  \\ \hline
\end{tabular}


\subsubsection{Functions}
\begin{itemize}
\item{\bf MFM\_LogInit}

This function initializes the log and should be called in a non-threaded
code section.

\begin{verbatim}
        function MFM_LogInit(name, logstate, labelio)
        character *(*) :: name              ! The name of the logfiles.
        integer :: logstate                 ! Beginning state of the log.
        logical :: labelio                  ! Separate or same file for MPI.
        type(MFM_LogHandle) :: MFM_LogInit  ! Returns a handle for this log
\end{verbatim}

This call will set up whatever data structures are necessary for logging, and
will open the logfiles as necessary.  The name parameter will be used to create
the log files.  If {\tt labelio} is {\tt .false.} then there will be one
logfile and the name will be exactly as specified.  If {\tt labelio} is {\tt
.true.}, then there will be one file per process with the name 
``{\tt name}.{\it \{MPI process number\}}''[ .

\end{itemize}
\subsubsection{Subroutines}
\begin{itemize}

\item{\bf MFM\_LogSetState}

Sets the log state.  The state is set at initialization, but this provides a
method to change the setting at any time.

\begin{verbatim}
        subroutine MFM_LogSetState(log, state)
        type(MFM_LogHandle) :: log      ! The log to set state in.
        integer :: state                   ! The new state.
\end{verbatim} 

\item{\bf MFM\_LogSelect}

Chooses the log to make active.  This log is used for subsequent calls 
to {\tt MFM\_LogPrint}.

\begin{verbatim}
        subroutine MFM_LogSelect(log)
        type(MFM_LogHandle) :: log      ! The log to use.
\end{verbatim} 

\item{\bf MFM\_LogPrint}

Outputs the desired message.

\begin{verbatim}
        
        subroutine MFM_LogPrint(LogLevel, format, ...)
        integer :: LogLevel      ! Logging level for this data.
        character(*) :: format   ! A C-Style format string        
\end{verbatim}

Example:

\begin{verbatim}

        real(8) :: eps
        real :: thet
        int :: i        

        call MFM_LogPrint(MFM_LOGLEVEL_INFO, "In Euler Transform, eps = %-10.3f, thet =
     &   %10.3lf, iteration:%d \n", eps, thet, i)

\end{verbatim}

The call will work exactly like {\tt printf} does in C.  This call will not
actually exist in this module since it accepts a variable argument list, and
this can only be done in C.  The function will appear to be included in this
module, however.

The log levels are defined above.

\item{\bf MFM\_LogFlush}
        
This function will not return until the log has flushed the output through
the output mechanism.

\end{itemize}
