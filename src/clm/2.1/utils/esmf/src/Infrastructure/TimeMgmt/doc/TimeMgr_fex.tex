% $Id$

The example below demonstrates usage of the {\tt TimeMgr} class.  For
a detailed description of the {\tt TimeMgr} class interface, see 
Appendix A.

We initialize a {\tt TimeMgr} with a start date, an end date, and a timestep.  
The {\tt TimeMgr} advances the model calendar until it recognizes that it is on the last 
timestep.  Since no memory is allocated from the heap when a {\tt TimeMgr} is initialized, 
there is no need to deallocate a {\tt TimeMgr} object.

\begin{verbatim}      

  use ESMF_TimeMgmtMod

  type(ESMF_Time) :: stepSize
  type(ESMF_Date) :: startDate, endDate
  type(ESMF_TimeMgr) :: timeMgr

!-------------------------------------------------------------------------------    
! Initialize the model timestep to 1 day, 1800 seconds; the start date to 
! December 1, 2001 and 0 seconds; and the end date to March 14, 2003 and 
! 0 seconds.  Dates are represented in a YYYYMMDD format.  In this example
! we use a Gregorian calendar. 
!-------------------------------------------------------------------------------

  stepSize = ESMF_TimeInit(1, 1800)
  startDate = ESMF_DateInit(ESMF_GREGORIAN, 20011201, 0)
  endDate = ESMF_DateInit(ESMF_GREGORIAN, 20030314, 0)

!-------------------------------------------------------------------------------    
! Initialize the time manager.
!-------------------------------------------------------------------------------

  timeMgr = ESMF_TimeMgrInit(stepSize, startDate, endDate)

!-------------------------------------------------------------------------------    
! Advance the model until the end date.
!-------------------------------------------------------------------------------

  do while (.NOT. ESMF_TimeMgrLastStep(timeMgr))
    ! Execute model code
    call ESMF_TimeMgrAdvance(timeMgr)
  end do

\end{verbatim}




