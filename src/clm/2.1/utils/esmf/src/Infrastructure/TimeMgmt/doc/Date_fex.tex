% $Id$

In the example below we demonstrate basic usage of the {\tt Date} class.  For
a detailed description of the {\tt Date} class interface, see 
Appendix A.

The example shows how to initialize a Gregorian {\tt Date} and increment 
it by a specified time interval.  Since no memory is allocated from the heap
when a {\tt Date} is initialized, there is no need to deallocate a 
{\tt Date} object.

\begin{verbatim}

  use ESMF_TimeMgmtMod

  type(ESMF_Time) :: interval
  type(ESMF_Date) :: startDate, endDate

!-------------------------------------------------------------------------------    
! Initialize a date to July 20, 1998 and 0 seconds. 
!-------------------------------------------------------------------------------

  startDate = ESMF_DateInit(ESMF_GREGORIAN, 19980720, 0) 

!-------------------------------------------------------------------------------    
! Initialize a time to 250 days and 7200 seconds. 
!-------------------------------------------------------------------------------

  interval = ESMF_TimeInit(250, 7200)

!-------------------------------------------------------------------------------    
! Increment the date with the time. 
!-------------------------------------------------------------------------------
       
  endDate = ESMF_DateIncrement(startDate, interval)
\end{verbatim}




